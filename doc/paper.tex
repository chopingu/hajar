% define class of document
\documentclass[titlepage]{article}

\hfuzz=50pt

% packages 
\usepackage[cmintegrals,cmbraces]{newtxmath}
\usepackage{ebgaramond-maths}
\usepackage[T1]{fontenc}
\usepackage{hyperref}

% keywords command
\providecommand{\keywords}[1] { 
    \small
    \textbf{\textit{Keywords ---}} #1
}

% title 
\title{\textbf{ANALYSIS OF REINFORCED \\NEURAL NETWORKS}}

% author and supervisors
\author{\textbf{\huge-----------------------------}\\[0.5in] \textit{Adam Amanbaev, Hugo Åkerfeldt, Jonathan Hallström, Romeo Patzer} \\\\
    \small \textit{Supervisors: Ulf Backlund \& Bertil Lundberg}
}

% date-specification
\date{}


% specify font of abstract-title
\renewcommand\abstractname{\textbf{Abstract \\[0.2in] \huge---------}}

% begin document
\begin{document}

% create the title and authors
\maketitle

% new page
\newpage

% abstract page
\begin{abstract}
    In late 2017 DeepMind announced a groundbreaking system in a preprint \cite{alphazero} and the results were astonishing. The system was called AlphaZero and utilized \emph{artificial neural networks} in order to teach itself the game chess without any proprietary knowledge, except the rules. After approximately 9 hours it was able to beat the strongest hand-crafted engines, such as Stockfish and it had learned centuries of human knowledge of chess. In this paper we aim to study the effectiveness of different \emph{neural networks} such as the one used in AlphaZero. To be precise, we will analyze the efficiency of those networks in combination with varying \emph{algorithms, optimizations, hyperparameters} and \emph {architectures} applied to the classic game and variations of connect-four. \\[0.5in]
\centerline{\keywords{Machine Learning, AI, Reinforcement Learning, Neural Network, Deep Learning}} 
\end{abstract}

% table of contents
\tableofcontents

% separate table of contents from sections
\newpage

% create sections 
\section{Introduction}

% subsection under nearest section above
\subsection{subsection}


% references
\begin{thebibliography}{100}
\bibitem{alphazero}
Silver, David; Hubert, Thomas; Schrittwieser, Julian; Antonoglou, Ioannis; Lai, Matthew; Guez, Arthur; Lanctot, Marc; Sifre, Laurent; Kumaran, Dharshan; Graepel, Thore; Lillicrap, Timothy; Simonyan, Karen; Hassabis, Demis (December 5, 2017). "Mastering Chess and Shogi by Self-Play with a General Reinforcement Learning Algorithm".

\end{thebibliography}

\end{document}
