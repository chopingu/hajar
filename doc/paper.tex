% define class of document
\documentclass[a4paper]{article}

% packages 
\usepackage{graphicx}
\usepackage[margin=25mm]{geometry}
\usepackage{amsmath}
\usepackage{amsfonts}
\usepackage{amssymb}
\usepackage{verbatim}
\usepackage{hyperref}
\pagenumbering{gobble}

% keywords command
\providecommand{\keywords}[1]
{ 
    \small
    \textbf{\textit{Keywords---}} #1
}


\title{An Analysis of Reinforced Neural Networks}
\author{Adam Amanbaev, Hugo Åkerfeldt, Jonathan Hallström, Romeo Patzer \\ 
        \small Danderyds Gymnasium \\
}

% \date{}

\begin{document}
\maketitle

\begin{abstract}
    In late 2017 DeepMind announced a groundbreaking system in a scientific paper \cite{alphazero} and the results were astonishing. The system was called AlphaZero and utilized \emph{artificial neural networks} in order to teach itself the game chess without any proprietary knowledge, except the rules. After approximately 9 hours it was able to beat the strongest hand-crafted engines, such as Stockfish and it had learned centuries of human knowledge of chess. In this paper we aim to study the effectiveness of different \emph{neural networks} such as the one used in AlphaZero. To be precise, we will analyze the efficiency of those networks in combination with varying \emph{algorithms, optimizations, hyperparameters} and \emph {architectures} applied to the classic game of connect four. 
\end{abstract} \hspace{10pt}

\keywords{Machine Learning, AI, Reinforcement Learning, Neural Network, Deep Learning}

\begin{thebibliography}{100}
\bibitem{alphazero}
Silver, David; Hubert, Thomas; Schrittwieser, Julian; Antonoglou, Ioannis; Lai, Matthew; Guez, Arthur; Lanctot, Marc; Sifre, Laurent; Kumaran, Dharshan; Graepel, Thore; Lillicrap, Timothy; Simonyan, Karen; Hassabis, Demis (December 5, 2017). "Mastering Chess and Shogi by Self-Play with a General Reinforcement Learning Algorithm".

\end{thebibliography}

\end{document}
