\documentclass[titlepage]{article}
\usepackage[cmintegrals,cmbraces]{newtxmath}
\usepackage{ebgaramond-maths}
\usepackage[T1]{fontenc}
\usepackage{hyperref}


\providecommand{\keywords}[1] {
    \small
    \textbf{\textit{Keywords ---}} #1
}

\title{
    \textbf{ANALYSIS OF \\
    MACHINE LEARNING \\
    APPLIED TO BOARD GAMES}
}

\author{
    \rule{8cm}{0.3mm} \\[0.5in] 
    \textit{Adam Amanbaev, Jonathan Hallström, Alvar Edvardsson} \\ 
    \textit{Hugo Åkerfeldt, Romeo Patzer} \\\\
    \small \textit{Supervisor: Ulf Backlund}
}

\date{}

\renewcommand\abstractname{\textbf{Abstract \\[0.1in] \rule{3cm}{0.2mm}}}

\renewcommand\contentsname{\Huge Contents \\}

\begin{document}

\maketitle

\newpage

% abstract page
\begin{abstract}
    In late 2017 DeepMind announced a groundbreaking system in a preprint \cite{alphazero} and the results were astonishing. The system was called AlphaZero and utilized \emph{artificial neural networks} in combination with \emph{heuristic algorithms} in order to teach itself the game chess without any proprietary knowledge. After approximately 9 hours it was able to beat the strongest hand-crafted engines, such as Stockfish and it had learned centuries of human knowledge of chess. In this paper we aim to study the effectiveness of different \emph{neural networks} and \emph{heuristic algorithms} such as the one used in AlphaZero. More precisely, we intend to analyze the efficiency of those networks and algorithms in combination with varying \emph{optimizations, parameters, hyperparameters} and \emph {architectures} applied to the classic games \emph{Connect Four} and \emph{Othello}. \\[0.5in]
\centerline{\keywords{Machine Learning, Supervised Learning, Reinforcement Learning, Neural Network, Deep Learning}}
\end{abstract}

\tableofcontents

\newpage

\section{Motivation}

\newpage

\section{Introduction}

\subsection{What is Reinforcement Learning}
\subsection{What is Deep Learning}
\subsubsection{Artificial Neural Networks}
\subsubsection{Deep Reinforcement Learning}

\newpage

\section{Notation and Definitions}
\subsection{Sigma Function}
\subsection{Vector}
\subsection{Matrix}
\subsection{Derivative}
\subsection{Gradient}

\newpage

\begin{thebibliography}{100}
\bibitem{alphazero}
Silver, David; Hubert, Thomas; Schrittwieser, Julian; Antonoglou, Ioannis; Lai, Matthew; Guez, Arthur; Lanctot, Marc; Sifre, Laurent; Kumaran, Dharshan; Graepel, Thore; Lillicrap, Timothy; Simonyan, Karen; Hassabis, Demis (December 5, 2017). "Mastering Chess and Shogi by Self-Play with a General Reinforcement Learning Algorithm".

\end{thebibliography}

\end{document}
